\documentclass[12pt]{article}

\begin{document}

\section{Introduction}

\section{Use Cases}
Besides the case where you explicitly have two objectives
that partially conflict with each other,
there are quite few situations where generating an efficient
frontier might apply.  
If you need to present 
alternative solutions, this is a good way to generate them.

If you have a solution that is technically optimal,

Gurobi can't directly optimize ratios, even if the
numerator and denominator are linear 
but it turns out, with some mild assumptions, 
you can generate an efficient frontier and inspect
it for the solution that maximizes of minimizes
that ratio. 

If you need to combine several goals and you can't
figure out how because either they can't all be put
into dollars, their relative importance isn't fixed
or know before hand, or if they have a non-linear
relationship.

It can also improve existing models.
If you have arbitrary coefficients in your objective
function that might be a sign that you are combining 
two criteria that might better be treated separately.
If you have wide variation in your coefficients
say you have coefficients in the single-digits
and in the millions
this can cause numerical issues.
If you see a message
``warning: switching to quad precision'' 
in your logs that's a possible sign that
you are trying to do too much with a single
objective.





\section{Outline}
In this screen cast, we will build and solve a simple linear
programming model with the Gurobi python API.  
The model we will use is the so-called transportation model
which is probably the simplest supply chain optimization
model that you can have.
Then we'll query the
attributes of the solved model.  We'll, not only look at the 'X'
attributes of your decision variables, which are the values that you
usually want, but also but also the attributes that show the
sensitivity of the solution to changes in the coefficients and bounds.
The most common in the dual price, which is also called shadow price
or bid price.  This is the incremental effect on the optimal objective
value of changing a bound on a constraint.  We are also going to look
at another family of attributes with the prefix SA which show you the
depth of the bid price.  That is, how much can a bound change before
the bid price itself changes.

Finally, we'll show the efficient frontier approach to working with 
two objective functions.  First we'll give the definition of Pareto 
optimality which is probably the most accepted
definition of optimality in the presence of two or more objective
functions and compare it with optimality with one objective
as defined by math programming solvers.

Then we'll compute all the Pareto efficient points for our simple
model, using two different approaches and graph the results.
We'll also relate the sensitivity attributes in Gurobi to the 

and demonstrate some how some of the efficient frontier
can be calculated just with the attributes available in Gurobi.




\end{document}
